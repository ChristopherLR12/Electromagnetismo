\documentclass{article}
\usepackage[utf8]{inputenc}
\usepackage{fancyhdr}
\rfoot{\thepage}
\setlength{\textwidth}{172mm}
\setlength{\textheight}{235mm}
\setlength{\oddsidemargin}{-7mm}
\setlength{\topmargin}{-20mm}

%%%%%%%%%%%%%%%%%%%%%%%%%%%%%%%%%%%%%%%%%%%%%%%%%%%%%%%%%%%%%%
\usepackage{amssymb}
\usepackage{amsmath}
\usepackage{rotating}
\usepackage{graphicx}
\graphicspath{{img/}}
\usepackage[mathscr]{euscript}
\usepackage[usenames,dvipsnames,svgnames,table]{xcolor}
\usepackage{multicol}
%%%%%%%%%%%%%%%%%%%%%%%%%%%%%%%%%%%%%%%%%%%%%%%%%%%%%%%%%%%%%%
%\input{definitions}
%\input{qmdef}
\newcommand{\be}{\begin{equation*}}
\newcommand{\ee}{\end{equation*}}
\newcommand{\ble}[1]{\begin{equation} \label{#1}}
\newcommand{\bae}{\begin{eqnarray}}
\newcommand{\eae}{\end{eqnarray}}
\newcommand{\sint}{\sin{\theta}}
\newcommand{\cost}{\cos{\theta}}
\newcommand{\pl}{\left(}
\newcommand{\pr}{\right)}
\newcommand{\kl}{\left[}
\newcommand{\kr}{\right]}
\newcommand{\ii}{\int_{-\infty}^\infty}
\providecommand{\abs}[1]{\lvert#1\rvert}
\providecommand{\norm}[1]{\lVert#1\rVert}
\newcommand{\pa}{\vspace{2mm}}
%%%%%%%%%%%%%%%%%%%%%%%%%%%%%%%%%%%%%%%%%%%%%%%%%%%%%%%%%%%%%%

\begin{document}

\centerline{\Large \textbf{\textsc{\textcolor{NavyBlue}{Óptica}}}}
\vspace{3mm}
\centerline{\large \textsc{TAREA 7}}
\vspace{2mm}
\centerline{\large \textsc{Christopher López Ruiz}}

\vspace{-13pt}

\begin{center}
\line(1,0){480}
\end{center}

%%%%%%%%%%%%%%%%%%%%%%%%%%%%%%%%%%%%%%%%%%%%%%%%%%%%%
%%%%                                                               Problemas                                                                %%%%
%%%%%%%%%%%%%%%%%%%%%%%%%%%%%%%%%%%%%%%%%%%%%%%%%%%%%

%%%%%%%%%%%%%%%%%%%%%%%%%%%%%%%%%%%%%%%%%%%%%%%%%%%%%

\vspace{3pt}
%%%%%%%%%%%%%%%%%%%%%%%  PROBLEMA 1  %%%%%%%%%%%%%%%%%%%%%%
%%%%%%%%%%%%%%%%%%%%%%%%%%%%%%%%%%%%%%%%%%%%%%%%%%%%%
%

\begin{description}
\item[\colorbox{TealBlue}{\textbf{1.-}}]

Se tiene que una interferencia constructiva se tiene

\be
y = \frac{L}{a}m\lambda \Rightarrow L = \frac{y\,a}{m \, \lambda}
\ee

Donde para la luz verde se tiene que $\lambda \approx 525 \, nm $, entonces sabiendo que $a$ es la distancia entre las rendijas, $y$ es la distancia de separación de las franjas de interferencia y tomando a $m = 1$ se tiene

\be
L = \frac{ 8 \times 10^{-4} \,m \, 25 \times 10^{-3} \, m}{1 \times \, 525 \times 10^{-9} \, m} = 38.09 \, m
\ee

Por lo tanto se necesita que la pantalla debe estar a $38.09 \, m$ de las rendijas.


\end{description}

\vspace{3pt}
%%%%%%%%%%%%%%%%%%%%%%%  PROBLEMA 2 %%%%%%%%%%%%%%%%%%%%%%
%%%%%%%%%%%%%%%%%%%%%%%%%%%%%%%%%%%%%%%%%%%%%%%%%%%%%\vspace{1mm}


\begin{description}
\item[\colorbox{TealBlue}{\textbf{2.-}}]

\end{description}

En el interferómetro de Michelson hay una interferencia constructiva cuando

\be
2d\,\cos{\theta} = \pl m + \frac{1}{2} \pr \lambda 
\ee

y el punto central es máximo cuando $\theta_0 = 0$, es decir cuando

\be
2d = \pl m_0 + \frac{1}{2} \pr \lambda \Rightarrow m_0 = \frac{2d}{\lambda} - \frac{1}{2}
\ee

Por otra parte la $N$-ésima franja del interferómetro está dada Por

\be
2d\,\cos{\theta_N} = \pl m_0 - N + \frac{1}{2} \pr \lambda \Rightarrow 
2d\,\cos{\theta_N} = \pl \frac{2d}{\lambda} - \frac{1}{2} - N + \frac{1}{2} \pr \lambda \Rightarrow 
\cos{\theta_N} = \pl \frac{1}{\lambda} - \frac{N}{2d} \pr \lambda
\ee

\be
\theta_N = \cos^{-1}\pl 1 - \frac{N\,\lambda}{2d} \pr
\ee

Tomando $\lambda = 605 \, nm = 605 \times 10^{-9} \, m$ entonces para el inciso $a)$

\be
\theta_0 = \cos^{-1}\pl 1 - \frac{10 \times 605 \times 10^{-9} \, m}{1.5 \times 10^{-3} \, m} \pr
\ee


\vspace{3pt}
%%%%%%%%%%%%%%%%%%%%%%%  PROBLEMA 3  %%%%%%%%%%%%%%%%%%%%%%
%%%%%%%%%%%%%%%%%%%%%%%%%%%%%%%%%%%%%%%%%%%%%%%%%%%%%

\begin{description}
\item[\colorbox{TealBlue}{\textbf{3.-}}]

\end{description}

\vspace{3pt}
%%%%%%%%%%%%%%%%%%%%%%%  PROBLEMA 4  %%%%%%%%%%%%%%%%%%%%%%
%%%%%%%%%%%%%%%%%%%%%%%%%%%%%%%%%%%%%%%%%%%%%%%%%%%%%

\begin{description}
\item[\colorbox{TealBlue}{\textbf{4.-}}]

Se tiene que para una interferencia constructiva

\be
\frac{x^2}{R} =\pl m+\frac{1}{2} \pr \lambda \Rightarrow R = \frac{x^2}{(m+\frac{1}{2}) \lambda}
\ee

Así entonces

\be
R = \frac{(1\times 10^{-2} \, m)^2}{(20+\frac{1}{2}) \, 500 \times 10^{-9} \, m} = 9.75 \, m
\ee

\end{description}

\vspace{3pt}

%%%%%%%%%%%%%%%%%%%%%%%  PROBLEMA 4  %%%%%%%%%%%%%%%%%%%%%%
%%%%%%%%%%%%%%%%%%%%%%%%%%%%%%%%%%%%%%%%%%%%%%%%%%%%%

\begin{description}
\item[\colorbox{TealBlue}{\textbf{5.-}}]
    
\end{description}
    
\vspace{3pt}

%%%%%%%%%%%%%%%%%%%%%%%%%%%%%%%%%%%%%%%%%%%%%%%%%%%

\begin{center}
\line(1,0){480}
\end{center}

\end{document}